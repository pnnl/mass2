\subsection{mass2gage.pl\label{mass2gage_pl}\index{mass2gage.pl}}


Extract data from a MASS2 gage output file (NetCDF format).

\subsection*{SYNOPSIS\label{mass2gage_pl_SYNOPSIS}\index{mass2gage pl!SYNOPSIS}}

perl \textbf{mass2gage.pl} \textbf{-l} \textit{file}



perl \textbf{mass2gage.pl} \textbf{-v} \textit{var} \textbf{-g} \textit{gage} [\textbf{-C}$|$\textbf{-M}] [\textbf{-1}]
[\textbf{-D}] \textbf{-X} \textit{n}] [\textbf{-o} \textit{output}] \textit{file}

\subsection*{DESCRIPTION\label{mass2gage_pl_DESCRIPTION}\index{mass2gage pl!DESCRIPTION}}

This script is used to extract data for one variable and gage from
a MASS2 gage output file.  It is primarily intended to be used to
check a simulation while in progress.

\subsection*{OPTIONS\label{mass2gage_pl_OPTIONS}\index{mass2gage pl!OPTIONS}}\begin{description}
\item[\textbf{-l}] \mbox{}

List the gages and time-dependant variables in \textit{file} and exit.

\item[\textbf{-v} \textit{var}] \mbox{}

(required) Extract the \textit{var} time-dependant variable from \textit{file};
\textit{var} may be either an integer or a variable name (as long as the
name does not start with a number), either of which can be obtained
using \textbf{-l}.

\item[\textbf{-g} \textit{gage}] \mbox{}

Extract data from the \textit{gage} location; \textit{gage} may be
either an integer or a gage location name (as long as the name does
not start with a number), either of which can be obtained using \textbf{-l}.

\item[\textbf{-C}] \mbox{}

Output a cumulative frequency distribution; normally, a time series is
output, this option will cause the data from the specified gage to be
sorted and assigned an exceedance probability.

\item[\textbf{-D}] \mbox{}

If the \textit{file} contains the variable \texttt{isdry}, extract records for
\textit{var} only when \texttt{isdry} is zero; applies only to the extraction of a
time series; a CFD extraction will still contain all values.

\item[\textbf{-o} \textit{output}] \mbox{}

Send extracted data to \textit{output} (does not work with \textbf{-l}).

\item[\textbf{-1}] \mbox{}

Add a line at the top of the output (line 1) containing some
information about the extracted data.

\item[\textbf{-M}] \mbox{}

Format as a MASS1/MASS2 boundary condition file (implies -1).

\item[\textbf{-X} \textit{n}] \mbox{}

Skip the first \textit{n} time records in the gage file, to avoid a warm up
period, for example.

\end{description}
\subsection*{EXAMPLES\label{mass2gage_pl_EXAMPLES}\index{mass2gage pl!EXAMPLES}}

This is the output from a listing (\textbf{-l}) of a particular gage.nc file:

\begin{verbatim}
    > \textbf{perl mass2gage.pl -l gage.nc }
    MASS2 Gage Output File:
             "gage.nc"
    1 time slices:
            starting: 03-19-1999 12:00:00
              ending: 03-19-1999 12:00:00
\end{verbatim}
\begin{verbatim}
    Available Gage Locations
    -----------------------------------------------------
    Gage Name                           Block   Eta    Xi
    -----------------------------------------------------
       1 T54LBC piezometer                  1   167    83
       2 T61IRC piezometer                  2    63    27
       3 T64RBC piezometer                  4   117     9
       4 T81RBC piezometer                 12    38    47
       5 T90RBC piezometer                 13   228    51
       6 T94RBC piezometer                 13   451    15
       7 T101LBC piezometer                14    24    63
       8 downstream extent                 16   155    40
       9 ERC 100-D                          1    22     7
      10 ERC 100-H                          4   115     9
      11 ERC 100-F                         10   116     9
      12 ERC 100-D Alternate                1    22    10
      13 trouble spot                      11     4    16
      14 trouble spot upstream             11     3    16
      15 trouble spot downstream           11     5    16
      16 trouble spot west                 11     4    15
      17 trouble spot east                 11     4    17
    -----------------------------------------------------
\end{verbatim}
\begin{verbatim}
    Available Time-Dependent Variables:
      8 wsel                 Water Surface Elevation, feet
      9 depth                Depth, feet
     10 vmag                 Velocity Magnitude, feet/second
     11 uvel                 Longitudinal Velocity, feet/second
     12 vvel                 Lateral Velocity, feet/second
     13 isdry                Dry Cell Flag, none
\end{verbatim}


Depth data is extracted from the same file:

\begin{verbatim}
    03-19-1999 12:00:00           8.369
    03-19-1999 13:00:00          9.9153
    03-19-1999 14:00:00          9.8377
    03-19-1999 15:00:00          9.7396
    03-19-1999 16:00:00          9.6508
    03-19-1999 17:00:00           9.562
    03-19-1999 18:00:00          9.4928
\end{verbatim}

%%% Local Variables: 
%%% mode: latex
%%% TeX-master: t
%%% End: 
